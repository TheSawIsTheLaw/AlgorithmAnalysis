\documentclass[12pt]{report}
\usepackage[utf8]{inputenc}
\usepackage[russian]{babel}
%\usepackage[14pt]{extsizes}
\usepackage{listings}
\usepackage[justification=centering]{caption}

\lstdefinelanguage{Kotlin}{
  comment=[l]{//},
  commentstyle={\color{gray}\ttfamily},
  emph={delegate, filter, first, firstOrNull, forEach, lazy, map, mapNotNull, println, return@},
  emphstyle={\color{OrangeRed}},
  identifierstyle=\color{black},
  keywords={abstract, actual, as, as?, break, by, class, companion, continue, data, do, dynamic, else, enum, expect, false, final, for, fun, get, if, import, in, interface, internal, is, null, object, override, package, private, public, return, set, super, suspend, this, throw, true, try, typealias, val, var, vararg, when, where, while},
  keywordstyle={\color{NavyBlue}\bfseries},
  morecomment=[s]{/*}{*/},
  morestring=[b]",
  morestring=[s]{"""*}{*"""},
  ndkeywords={@Deprecated, @JvmField, @JvmName, @JvmOverloads, @JvmStatic, @JvmSynthetic, Array, Byte, Double, Float, Int, Integer, Iterable, Long, Runnable, Short, String},
  ndkeywordstyle={\color{BurntOrange}\bfseries},
  sensitive=true,
  stringstyle={\color{ForestGreen}\ttfamily},
}

% Для листинга кода:
\lstset{ %
language=Kotlin,                 % выбор языка для подсветки (здесь это С)
basicstyle=\small\sffamily, % размер и начертание шрифта для подсветки кода
numbers=left,               % где поставить нумерацию строк (слева\справа)
numberstyle=\tiny,           % размер шрифта для номеров строк
stepnumber=1,                   % размер шага между двумя номерами строк
numbersep=5pt,                % как далеко отстоят номера строк от подсвечиваемого кода
showspaces=false,            % показывать или нет пробелfile:///home/trvehazzk3r/Desktop/test/report.teы специальными отступами
showstringspaces=false,      % показывать или нет пробелы в строках
showtabs=false,             % показывать или нет табуляцию в строках
frame=single,              % рисовать рамку вокруг кода
tabsize=2,                 % размер табуляции по умолчанию равен 2 пробелам
captionpos=t,              % позиция заголовка вверху [t] или внизу [b] 
breaklines=true,           % автоматически переносить строки (да\нет)
breakatwhitespace=false, % переносить строки только если есть пробел
escapeinside={\#*}{*)}   % если нужно добавить комментарии в коде
}

\usepackage{hyperref}
\hypersetup{
    linktoc=all,     %set to all if you want both sections and subsections linked
    linkcolor=blue,  %choose some color if you want links to stand out
}

% Для измененных титулов глав:
\usepackage{titlesec, blindtext, color} % подключаем нужные пакеты
\definecolor{gray75}{gray}{0.75} % определяем цвет
\newcommand{\hsp}{\hspace{20pt}} % длина линии в 20pt
% titleformat определяет стиль
\titleformat{\chapter}[hang]{\Huge\bfseries}{\thechapter\hsp\textcolor{gray75}{|}\hsp}{0pt}{\Huge\bfseries}


% plot
\usepackage{pgfplots}
\usepackage{filecontents}
\usetikzlibrary{datavisualization}
\usetikzlibrary{datavisualization.formats.functions}
\begin{filecontents}{LevRec.dat}
5 10867
7 258961
10 33589820
\end{filecontents}

\begin{filecontents}{LevRecMat.dat}
10 3146
20 12896
30 29325
50 70918
100 184238
200 643895
\end{filecontents}

\begin{filecontents}{LevIterMat.dat}
10 2001
20 4686
30 10744
50 29257
100 86268
200 248651
\end{filecontents}

\begin{filecontents}{DamLev.dat}
10 2137
20 6251
30 13631
50 38427
100 118891
200 299743
\end{filecontents}


\begin{document}
\begin{titlepage}
	\fontsize{12pt}{12pt}\selectfont
	\noindent \begin{minipage}{0.15\textwidth}
		\includegraphics[width=\linewidth]{inc/img/b_logo.jpg}
	\end{minipage}
	\noindent\begin{minipage}{0.9\textwidth}\centering
		\textbf{Министерство науки и высшего образования Российской Федерации}\\
		\textbf{Федеральное государственное бюджетное образовательное учреждение высшего образования}\\
		\textbf{«Московский государственный технический университет имени Н.Э.~Баумана}\\
		\textbf{(национальный исследовательский университет)»}\\
		\textbf{(МГТУ им. Н.Э.~Баумана)}
	\end{minipage}
	
	\noindent\rule{15cm}{3pt}
	\newline\newline
	\noindent ФАКУЛЬТЕТ \underline{~~~~~~~~~~~~~~~~«Информатика и системы управления»~~~~~~~~~~~~~~~~} \newline\newline
	\noindent КАФЕДРА \underline{«Программное обеспечение ЭВМ и информационные технологии»}\newline\newline\newline\newline\newline\newline\newline
	
	
	\begin{center}
		\Large\textbf{Отчет по лабораторной работе №1 по курсу "Анализ алгоритмов"}\newline
	\end{center}
	
	\noindent\textbf{Тема} \underline{Расстояния Левенштейна и Дамерау-Левенштейна}\newline\newline\newline
	\noindent\textbf{Студент} \underline{Якуба Д. В.}\newline\newline
	\noindent\textbf{Группа} \underline{ИУ7-53Б}\newline\newline
	\noindent\textbf{Оценка (баллы)} \underline{~~~~~~~~~~~~~~~~~~~}\newline\newline
	\noindent\textbf{Преподаватели} \underline{Волкова Л.Л., Строганов Ю.В.}\newline
	
	\begin{center}
		\vfill
		Москва~---~\the\year
		~г.
	\end{center}
\end{titlepage}

\tableofcontents

\newpage
\chapter*{Введение}
\addcontentsline{toc}{chapter}{Введение}
\section*{Цели лабораторной работы}
\begin{itemize}
\item изучение классического алгоритма, алгоритма Копперсмита-Винограда и оптимизированного алгоритма Копперсмита-Винограда умножения матриц;
\item реализация классического алгоритма, алгоритма Копперсмита-Винограда и оптимизиованного алгоритма Копперсмита-Винограда умножения матриц;
\end{itemize}
\section*{Определение}
Алгоритм Копперсмита-Винограда = это алгоритм умножения квадратных матриц, предложенный в 1987 году Д. Копперсмитом и Ш. Виноградом \cite{CoppersmithAndWinograd}. В исходной весрии асимптотическая сложность алгоритма составляла $O(n^{2.3755})$, где $n$ 0 размер стороны матрицы. Алгоритм Копперсмита-Винограда с учётом усерии улучшений и доработок в последующие годы, обладает лучшей асимптотикой среди известных алгоритмов умножения матриц

\chapter{Аналитическая часть}


\section{Рекурсивный алгоритм нахождения расстояния Левенштейна}

\section*{Вывод}

\newpage

\chapter{Конструкторская часть}
\section{Блок-схема рекурсивного алгоритма Левенштейна}


\chapter{Технологическая часть}
\section{Требования к программному обеспечению}

\section{Средства реализации программного обеспечения}
При написании программного продукта был использован язык программирования Kotlin \cite{Kotlin}.

Данный выбор обусловлен следующими факторами:
\begin{itemize}
\item Высокая вычислительная производительность;.
\end{itemize}

Для тестирования производительности реализаций алгоритмов использовалась утилита measureNanoTime.

При написаннии программного продукта использовалась среда разработки IntelliJ IDEA.

Данный выбор обусловлен тем, что язык программирования Kotlin - это разработка компании JetBrains, поставляющей данную среду разработки;

\section{Листинг кода}
В листингах \ref{levRecList} - \ref{DamLevList} предоставлены реализации рассматриваемых алгоритмов.
\begin{lstlisting}[caption=Функция реализации рекурсивного алгоритма Левенштейна,
label={levRecList}]
s
\end{lstlisting}

\begin{lstlisting}[caption=Функция реализации рекурсивного алгоритма Левенштейна с использованием матрицы расстояний,
label={levRecMatList}]

\end{lstlisting}

\begin{lstlisting}[caption=Функция реализации итеративного алгоритма Левенштейна,
label={levIterList}]

\end{lstlisting}

\begin{lstlisting}[caption=Функция реализации алгоритма Дамерау-Левенштейна,
label={DamLevList}]

\end{lstlisting}

\section{Тестирование программного продукта}
В таблице \ref{table} приведены тестовые данные и вывод программы для алгоритмов вычисления расстояния Левенштейна и Дамерау-Левенштейна. Тесты пройдены успешно.

\newpage
\begin{table}[h]
	\begin{center}
		\caption{\label{table} Тесты}
		\begin{tabular}{|c|c|c|c|}
	\hline
			                    &                    & \multicolumn{2}{c|}{\bfseries Ожидаемый результат}    \\ \cline{3-4}\hline
	Строка 1& Строка 2 & Алг. Левенштейна & Алг. Дамерау-Левенштейна \\ [0.5ex] 
 	\hline\hline
 	table & tumbler & 3 & 3\\
 	\hline
 	hell & help & 1 & 1\\
 	\hline
	KillUsAll & KlilUsAll & 2 & 1\\
	\hline
	smoke & mssql & 5 & 4\\
	\hline
	OfMiceAndMen & OfMonstersAndMen & 6 & 6\\
	\hline
	roofer & killer & 4 & 4\\
	\hline
	orange & orangina & 3 & 3\\
	\hline
	prolifer & profiler & 2 & 2\\
	\hline
	cat & dog & 3 & 3\\
	\hline
		\end{tabular}
	\end{center}
\end{table}

\section*{Вывод}
Спроектированные алгоритмы вычисления расстояния Левенштейна рекурсивно, рекурсивно с использованием матрицы расстояний, итеративно с использованием матрицы расстояний, а также алгоритм вычисления расстояния Дамерау-Левенштейна итеративно с использованием матрицы были реализованы и протестированы.

\chapter{Исследовательская часть}

\section{Пример работы программного обеспечения}


\section{Технические характеристики}
Технические характеристики ЭВМ, на котором выполнялись исследования:
\begin{itemize}
\item ОС: Manjaro Linux 20.1.1 Mikah
\item Оперативная память: 16 Гб
\item Процессор: Intel Core i7-10510U

При проведении замеров времени ноутбук был подключен к сети электропитания.
\end{itemize}

\section{Время выполнения алгоритмов}
Алгоритмы тестировались на данных, сгенерированных случайным образом один раз.

Тестовые данные:
\begin{itemize}
\item 5 символов:\newline
VxgtU (строка 1),\newline
jRMFA (строка 2)
\item 7 символов:\newline
VxgtUsx (строка 1),\newline
jRMFAyC (строка 2)
\item 10 символов:\newline
VxgtUsx2u3 (строка 1),\newline
jRMFAyCfiV (строка 2)
\item 20 символов:\newline
VxgtUsx2u39dtX81sxy8 (строка 1),\newline
jRMFAyCfiVxyhmILtGMG (строка 2)
\item 30 символов:\newline
VxgtUsx2u39dtX81sxy8GInrYeVNmJ (строка 1),\newline
jRMFAyCfiVxyhmILtGMG4IVZTjPQ7l (строка 2)
\item 50 символов:\newline
VxgtUsx2u39dtX81sxy8GInrYeVNmJvvG7WkaA7Qjs82qP6bJG (строка 1),\newline jRMFAyCfiVxyhmILtGMG4IVZTjPQ7laMIEG6xv9zbdXq9WcJY2 (строка 2)
\item 100 символов:\newline
VxgtUsx2u39dtX81sxy8GInrYeVNmJvvG7WkaA7Qjs82qP6bJG\newline
Ooryez5fYpJWcPRhm7TEjeUoD49M26XDt CJrGtjJXf3aZ9La9n (строка 1),\newline jRMFAyCfiVxyhmILtGMG4IVZTjPQ7laMIEG6xv9zbdXq9WcJY2\newline
G4J0JV1XP8ecmHkTYdY1uzSm8WFY3KjgG ggAw3GrPISl76Mzb1 (строка 2)
\item 200 символов:\newline
VxgtUsx2u39dtX81sxy8GInrYeVNmJvvG7WkaA7Qjs82qP6bJGO\newline
oryez5fYpJWcPRhm7TEjeUoD49M26XDtCJrGtjJXf3aZ9La9nsh\newline
v3cAbwuAJuKc00ndp6EWNHQcArjwXQzAtdpnHs2uOF1kfhWjzXU\newline
S44zKnHVNCaeLyzBlce3RCdGwbJx8s2SlfvYoyBZsKrN1cX (строка 1),\newline
jRMFAyCfiVxyhmILtGMG4IVZTjPQ7laMIEG6xv9zbdXq9WcJY2G\newline
4J0JV1XP8ecmHkTYdY1uzSm8WFY3KjgGggAw3GrPISl76Mzb1f3\newline
ElDEyOeorQGS6CxLWS3lH8sNgZta9vSDMLvnbPaXP24H5dYkBXL\newline
RruvzSlLs1T8hyezy0U3awz65ctATEclCBG4H1pC9mMusWF (строка 2)
\end{itemize}

Результаты замеров времени приведены в таблице \ref{time}. На рисунках \ref{recTime}, \ref{IterTime} приведены графики зависимостей времени работы алгоритмов от //.

\newpage
\begin{table}[h]
	\begin{center}
		\caption{\label{time} Замеры времени для строк различной длины}
		\begin{tabular}{|c c c c c|} 
 			\hline
			Длина строк & LevRec & LevMatRec & LevMatIter & DamLev \\ [0.5ex] 
 			\hline\hline
 			5 & 10867 & - & - & -\\
 			\hline
 			7 & 258961 & - & - & -\\
 			\hline
			10 & 33589820 & 3146 & 2001 & 2137\\
			\hline
			20 & - & 12896 & 4686 & 6251\\
			\hline
			30 & - & 29325 & 10744 & 13631\\
			\hline
			50 & - & 70918 & 29277 & 38427\\
			\hline
			100 & - & 184238 & 86268 & 118891\\
			\hline
			200 & - & 642895 & 248651 & 299743\\
			\hline
			\end{tabular}
	\end{center}
\end{table}

\begin{figure}[h]
\begin{center}
	\begin{tikzpicture}

\begin{axis}[
  		  	axis lines = left,
  		  	xlabel = $len$,
  		  	ylabel = {$time$},
			legend pos=north west,
			ymajorgrids=true
		] 
		\addplot[color=orange] table[x index=0, y index=1] {LevRecMat.dat};
		\addplot[color=blue, mark=square] table[x index=0, y index=1] {LevIterMat.dat};

		\addlegendentry{Рекурсивная реализация с матрицей}
		\addlegendentry{Итеративная реализация с матрицей}
		\end{axis}
	\end{tikzpicture}
	\captionsetup{justification=centering}
	\caption{Зависимость времени работы рекурсивных реализаций алгоритмов вычисления расстояния Левенштейна от длины строк}
	\label{recTime}
	\end{center}
\end{figure}

\begin{figure}[h]
	\begin{center}
	\begin{tikzpicture}

		\begin{axis}[
 		   	axis lines = left,
 		   	xlabel = $len$,
 		   	ylabel = {$time$},
			legend pos=north west,
			ymajorgrids=true
		]
		\addplot[color=blue, mark=square] table[x index=0, y index=1] {LevIterMat.dat};
		\addplot[color=green, mark=square] table[x index=0, y index=1] {DamLev.dat};

		\addlegendentry{Левенштейн}
		\addlegendentry{Дамерау-Левенштейн}
		\end{axis}
\end{tikzpicture}
	\captionsetup{justification=centering}
	\caption{Зависимость времени работы итеративных реализаций алгоритмов вычисления расстояния Левенштейна и Дамерау-Левенштейна от длины строк.}
	\label{IterTime}
	\end{center}
\end{figure}

\section{Оценка затрат памяти}
Максимальная глубина стека вызовов при исполнении рекурсивного алгоритма Левенштейна определяется выражением \ref{quoteForRec}:

\begin{equation}
\label{quoteForRec}
(sizeof(s_{1}) + sizeof(s_{2})) * (2 * sizeof(string) + sizeof(int))
\end{equation}
Здесь sizeof - оператор вычисления размера; $s_{1}$, $s_{2}$ - строки; $string$ - строковый тип; $int$ - целочисленный тип.

При исполнении интеративной реализации задействованная память будет определяться выражением \ref{quoteForIter}:

\begin{equation}
\label{quoteForIter}
(sizeof(s_{1} + 1) * (sizeof(s_{2} + 1) * sizeof(int) + sizeof(int) + 2 * sizeof(string)
\end{equation}

\section*{Вывод}

\chapter*{Заключение}
\addcontentsline{toc}{chapter}{Заключение}
В ходе выполнения лабораторной работы:
\begin{itemize}
\item 
\item 
\item 
\item 
\item 
\item Были получены практические навыки реализации алгоритмов на ЯП Kotlin.
\end{itemize}

\chapter*{Список источников}
\addcontentsline{toc}{chapter}{Список источников}
\bibliographystyle{utf8gost705u}  % стилевой файл для оформления по ГОСТу
\bibliography{biblio.bib}          % имя библиографической базы (bib-файла)


\end{document} 
