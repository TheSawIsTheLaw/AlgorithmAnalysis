\documentclass[12pt]{report}
\usepackage[utf8]{inputenc}
\usepackage[russian]{babel}
%\usepackage[14pt]{extsizes}
\usepackage{listings}
\usepackage{amsmath}
\usepackage[justification=centering]{caption}

% Для листинга кода:
\lstset{ %
language=c,                 % выбор языка для подсветки (здесь это С)
basicstyle=\footnotesize\sffamily, % размер и начертание шрифта для подсветки кода
numbers=left,               % где поставить нумерацию строк (слева\справа)
numberstyle=\tiny,           % размер шрифта для номеров строк
stepnumber=1,                   % размер шага между двумя номерами строк
numbersep=5pt,                % как далеко отстоят номера строк от подсвечиваемого кода
showspaces=false,            % показывать или нет пробелы специальными отступами
showstringspaces=false,      % показывать или нет пробелы в строках
showtabs=false,             % показывать или нет табуляцию в строках
frame=single,              % рисовать рамку вокруг кода
tabsize=2,                 % размер табуляции по умолчанию равен 2 пробелам
captionpos=t,              % позиция заголовка вверху [t] или внизу [b] 
breaklines=true,           % автоматически переносить строки (да\нет)
breakatwhitespace=false, % переносить строки только если есть пробел
escapeinside={\#*}{*)}   % если нужно добавить комментарии в коде
}

\usepackage{hyperref}
\hypersetup{
    linktoc=all,     %set to all if you want both sections and subsections linked
    linkcolor=blue,  %choose some color if you want links to stand out
}

% Для измененных титулов глав:
\usepackage{titlesec, blindtext, color} % подключаем нужные пакеты
\definecolor{gray75}{gray}{0.75} % определяем цвет
\newcommand{\hsp}{\hspace{20pt}} % длина линии в 20pt
% titleformat определяет стиль
\titleformat{\chapter}[hang]{\Huge\bfseries}{\thechapter\hsp\textcolor{gray75}{|}\hsp}{0pt}{\Huge\bfseries}

% plot
\usepackage{pgfplots}
\usepackage{filecontents}
\usetikzlibrary{datavisualization}
\usetikzlibrary{datavisualization.formats.functions}

\begin{document}
\begin{titlepage}
	\fontsize{12pt}{12pt}\selectfont
	\noindent \begin{minipage}{0.15\textwidth}
		\includegraphics[width=\linewidth]{inc/img/b_logo.jpg}
	\end{minipage}
	\noindent\begin{minipage}{0.9\textwidth}\centering
		\textbf{Министерство науки и высшего образования Российской Федерации}\\
		\textbf{Федеральное государственное бюджетное образовательное учреждение высшего образования}\\
		\textbf{«Московский государственный технический университет имени Н.Э.~Баумана}\\
		\textbf{(национальный исследовательский университет)»}\\
		\textbf{(МГТУ им. Н.Э.~Баумана)}
	\end{minipage}
	
	\noindent\rule{15cm}{3pt}
	\newline\newline
	\noindent ФАКУЛЬТЕТ \underline{~~~~~~~~~~~~~~~~«Информатика и системы управления»~~~~~~~~~~~~~~~~} \newline\newline
	\noindent КАФЕДРА \underline{«Программное обеспечение ЭВМ и информационные технологии»}\newline\newline\newline\newline\newline\newline\newline
	
	
	\begin{center}
		\Large\textbf{Отчет по лабораторной работе №6 по курсу "Анализ алгоритмов"}\newline
	\end{center}
	
	\noindent\textbf{Тема} \underline{Муравьиный алгоритм}\newline\newline\newline
	\noindent\textbf{Студент} \underline{Якуба Д. В.}\newline\newline
	\noindent\textbf{Группа} \underline{ИУ7-53Б}\newline\newline
	\noindent\textbf{Оценка (баллы)} \underline{~~~~~~~~~~~~~~~~~~~}\newline\newline
	\noindent\textbf{Преподаватели} \underline{Волкова Л.Л., Строганов Ю.В.}\newline
	
	\begin{center}
		\vfill
		Москва~---~\the\year
		~г.
	\end{center}
\end{titlepage}

\setcounter{page}{2}

\tableofcontents

\newpage
\chapter*{Введение}
\addcontentsline{toc}{chapter}{Введение}
\section*{Цель лабораторной работы}
Реализация муравьиного алгоритма и приобретение навыков параметризации методов на примере реализованного алгоритма, примененного к задаче коммивояжера.
\section*{Задачи лабораторной работы}
\begin{enumerate}
\item[1)] изучить алгоритм полного перебора для решения задачи коммивояжера;
\item[2)] реализовать алгоритм полного перебора для решения задачи коммивояжера;
\item[3)] изучить муравьиный алгоритм для решения задачи коммивояжера;
\item[4)] реализовать муравьиный алгоритм для решения задачи коммивояжера;
\item[5)] провести сравнительный анализ скорости работы реализованных алгоритмов;
\item[5)] подготовить отчёт по проведенной работе.
\end{enumerate}

\chapter{Аналитическая часть}
В данном разделе описаны задача коммивояжёра, идея муравьиного алгоритма и алгоритма полного перебора для решения этой задачи.

\section{Задача коммивояжера}
Коммивояжёр (фр. commis voyageur) — бродячий торговец. Задача коммивояжёра — важная задача транспортной логистики, отрасли, занимающейся планированием транспортных перевозок \cite{Commie}. В описываемой задаче рассматривается несколько городов  и матрица попарных расстояний между ними. Требуется найти такой порядок  посещения  городов,  чтобы  суммарное  пройденное  расстояние было минимальным, каждый город посещался ровно один раз и  коммивояжер  вернулся  в  тот  город,  с  которого  начал  свой  маршрут.  Другими  словами,  во  взвешенном  полном  графе  требуется 
найти гамильтонов цикл минимального веса.

\section{Алгоритм полного перебора для решения задачи коммивояжера}
Алгоритм полного перебора для решения задачи коммивояжера предполагает рассмотрение всех возможных путей в графе и выбор наименьшего из них.

Такой подход гарантирует точное решение задачи, однако, так как задача относится к числу трансвычислительных \cite{Trans}, то уже при небольшом числе городов решение за приемлемое время невозможно.

\section{Муравьиный алгоритм для решения задачи коммивояжера}
Муравьиные алгоритмы представляют собой новый перспективный метод решения задач оптимизации, в основе которого лежит моделирование поведения колонии муравьев \cite{Ulianov}. Колония представляет собой систему с очень простыми правилами автономного поведения особей.

Каждый муравей определяет для себя маршрут, который необходимо пройти на основе феромона, который он ощущает, во время прохождения, каждый муравей оставляет феромон на своем пути, чтобы остальные муравьи могли по нему ориентироваться. В результате при прохождении каждым муравьем различного маршрута наибольшее число феромона остается на оптимальном пути. 

Самоорганизация колонии является результатом взаимодействия следующих компонентов:
\begin{itemize}
	\item случайность — муравьи имеют случайную природу движения;
	\item многократность — колония допускает число муравьев, достигающее от нескольких десятков до миллионов особей;
	\item положительная обратная связь — во время движения муравей откладывает феромон, позволяющий другим особям определить для себя оптимальный маршрут;
	\item отрицательная обратная связь — по истечении определенного времени феромон испаряется;
	\item целевая функция.
\end{itemize}

Пусть муравей обладает следующими характеристиками:
\begin{itemize}
	\item зрение — определяет длину ребра;
	\item обоняние — чувствует феромон;
	\item память — запоминает маршрут, который прошел.
\end{itemize}

Введем целевую функцию $\eta_{ij} = 1 / D_{ij}$, где $D_{ij}$ — расстояние из текущего пункта $i$ до заданного пункта $j$.

Посчитаем вероятности перехода в заданную точку по формуле \eqref{possibility}:
\begin{equation}
	\label{possibility}
	P_{kij} = \begin{cases}
		\frac{t_{ij}^a\eta_{ij}^b}{\sum_{q=1}^m t^a_{iq}\eta^b_{iq}}, \textrm{вершина не была посещена ранее муравьем k,} \\
		0, \textrm{иначе}
	\end{cases}
\end{equation}
где $a, b$ -- настраиваемые параметры, $t$ - концентрация феромона, причем $a + b = const$, а при $a = 0$ алгоритм вырождается в жадный \cite{Levitin}.

Когда все муравьи завершили движение происходит обновление феромона по формуле \eqref{pheromone1}:
\begin{equation}
	\label{pheromone1}
	t_{ij}(t+1) = (1-p)t_{ij}(t) + \Delta t_{ij}, \Delta t_{ij} = \sum_{k=1}^N t^k_{ij}
\end{equation}
где
\begin{equation}
	\label{pheromone2}
	\Delta t^k_{ij} = \begin{cases}
		Q/L_{k}, \textrm{ребро посещено k-ым муравьем,} \\
		0, \textrm{иначе}
	\end{cases}
\end{equation}
$L_{k}$ — длина пути k-ого муравья, $Q$ — настраивает концентрацию нанесения/испарения феромона, $N$ — количество муравьев.

\section*{Вывод}
Были рассмотрены задача коммивояжера, муравьиный алгоритм и алгоритм полного перебора для решения поставленной задачи.

В данной работе стоит задача реализации двух рассмотренных алгоритмов.

\chapter{Конструкторская часть}
В данном разделе представлены схемы муравьиного алгоритма и алгоритма полного перебора для решения задачи коммивояжера.
\section{Схема алгоритмы полного перебора}
Схема алгоритма Брезенхема предоставлена на рисунках \ref{img:bresAlg}, \ref{img:bresAlgPart}.

\begin{figure}
\begin{center}
\includegraphics[scale=0.4]{inc/img/bresAlg.png}
\captionsetup{justification=centering}
	\caption{Схема алгоритма Брезенхема.}
	\label{img:bresAlg}	
\end{center}
\end{figure}

\begin{figure}
\begin{center}
\includegraphics[scale=0.4]{inc/img/bresAlgPart.png}
\captionsetup{justification=centering}
	\caption{Схема алгоритма Брезенхема.}
	\label{img:bresAlgPart}	
\end{center}
\end{figure}

\newpage

\section{Схема муравьиного алгоритма}
На рисунках \ref{img:conveyor}, \ref{img:conveyor:2} предоставлена IDEF0 схема реализации конвейерной обработки данных для алгоритма Брезенхема.

Как видно из схемы уровня декомпозиции 1 (\ref{img:conveyor:2}) задействуется 3 потока. Поток №1 решает задачу подготовки данных для растрирования отрезка. Поток №2, задействуя входные данные и данные, полученные от потока №1, заполняет массив точек, наилучшим образом аппроксимирующих отрезок целочисленными координатами экрана пользователя. Поток №3, опираясь на данные, полученные от потока №2, непосредственно создает объект, который и будет в дальнейшем визуализирован.

\begin{figure}[ht]
\begin{center}
\includegraphics[scale=0.25]{inc/img/ramus/01_A0.png}
\captionsetup{justification=centering}
	\caption{Схема реализации конвейерной обработки данных для алгоритма Брезенхема.}
	\label{img:conveyor}	
\end{center}
\end{figure}

\newpage

\begin{figure}[ht]
\begin{center}
\includegraphics[scale=0.25]{inc/img/ramus/03_A1.png}
\captionsetup{justification=centering}
	\caption{Схема реализации конвейерной обработки данных для алгоритма Брезенхема.}
	\label{img:conveyor:2}	
\end{center}
\end{figure}

\newpage

\section*{Вывод}
Были представлены схемы алгоритма Брезенхема, а также реализации конвейерной обработки данных для данного алгоритма.

\chapter{Технологическая часть}
В данном разделе приведены требования к программному обеспечению, средства реализации программного обеспечения, а также листинг кода.

\section{Требования к программному обеспечению}
\begin{itemize}
\item входные данные - количество выполняемых задач (количество растеризуемых отрезков);
\item выходные данные - записи времени прихода и ухода обрабатываемых заявок для каждого реализованного конвейера.
\end{itemize}

\section{Средства реализации программного обеспечения}
При написании программного продукта был использован язык программирования C++ \cite{Cpp}.

Данный выбор обусловлен следующими факторами:
\begin{itemize}
\item данный язык программирования преподавался в рамках курса объектно-ориентированного программирования;
\item высокая вычислительная производительность;
\item большое количество справочной и учебной литературы в сети Интернет;
\item наличие реализации нативных потоков.
\end{itemize}

При написании программного продукта использовалась среда разработки QT Creator \cite{QT}.

Данный выбор обусловлен следующими факторами:
\begin{itemize}
\item основы работы с данной средой разработки преподавался в рамках курса программирования на Си;
\item QT Creator позволяет работать с расширением QtDesign, позволяющим создавать визуализируемый объект.
\end{itemize}

Для проведения замеров времени использовалась сторонняя библиотека Boost \cite{Boost}. Данная библиотека позволила фиксировать время прихода и ухода каждой заявки с точностью до наносекунд.

\section{Листинг кода}
В листингах \ref{list:bresAlg} и \ref{list:director} предоставлены реализации рассматриваемых алгоритмов.
\begin{lstlisting}[caption=Разбиение алгоритма Брезенхема,
label={list:bresAlg}]
std::string now_str()
{
    const boost::posix_time::ptime now = boost::posix_time::microsec_clock::local_time();

    const boost::posix_time::time_duration td = now.time_of_day();

    const long hours = td.hours();
    const long minutes = td.minutes();
    const long seconds = td.seconds();
    const long nanoseconds =
        td.total_nanoseconds() - ((hours * 3600 + minutes * 60 + seconds) * 1000000000);

    char buf[40];
    sprintf(buf, "%02ld:%02ld:%02ld.%09ld", hours, minutes, seconds, nanoseconds);

    return buf;
}

SegmentRasterizator::SegmentRasterizator(int xStart_, int yStart_, int xEnd_, int yEnd_)
{
    xStart = xStart_;
    yStart = yStart_;
    xEnd = xEnd_;
    yEnd = yEnd_;
    if (xStart == xEnd)
        xEnd += 1;
    else if (yStart == yEnd)
        yEnd += 1;

    image = new QImage(WIDTH, HEIGHT, QImage::Format_RGB32);
    image->fill(Qt::white);
}

int sign(float num)
{
    return (num < -__FLT_EPSILON__) ? -1 : ((num > __FLT_EPSILON__) ? 1 : 0);
}

void SegmentRasterizator::prepareConstantsForRB(int index)
{
    std::printf(ANSI_BLUE_BRIGHT "From START worker: task %d BEGIN %s" ANSI_RESET "\n", index, now_str().c_str());

    deltaX = xEnd - xStart;
    deltaY = yEnd - yStart;

    stepX = sign(deltaX);
    stepY = sign(deltaY);

    deltaX = std::abs(deltaX);
    deltaY = std::abs(deltaY);

    if (deltaX < deltaY)
    {
        std::swap(deltaX, deltaY);
        stepFlag = true;
    }
    else
        stepFlag = false;

    tngModule = deltaY / deltaX;
    mistake = tngModule - 0.5;

    std::printf(ANSI_BLUE_BRIGHT "From START worker: task %d ENDED %s" ANSI_RESET "\n", index, now_str().c_str());
}

void SegmentRasterizator::rastSegment(int index)
{
    std::printf(ANSI_MAGENTA_BRIGHT "From MIDDLE worker: task %d BEGIN %s" ANSI_RESET "\n", index, now_str().c_str());

    float curX = xStart, curY = yStart;
    for (int i = 0; i <= deltaX; i++)
    {
        dotsOfSegment.push_back(std::pair<int, int>(curX, curY));
        if (stepFlag)
        {
            if (mistake >= 0)
                (curX += stepX, mistake--);
            curY += stepY;
        }
        else
        {
            if (mistake >= 0)
                (curY += stepY, mistake--);
            curX += stepX;
        }
        mistake += tngModule;
    }

    std::printf(ANSI_MAGENTA_BRIGHT "From MIDDLE worker: task %d ENDED %s" ANSI_RESET "\n", index, now_str().c_str());
}

void SegmentRasterizator::createImg(int index)
{
    std::printf(ANSI_CYAN_BRIGHT"From END worker: task %d BEGIN %s" ANSI_RESET "\n", index, now_str().c_str());

    for (auto iter = dotsOfSegment.begin(); iter < dotsOfSegment.end(); iter++)
        image->setPixel(iter->first, iter->second, Qt::black);

    std::printf(ANSI_CYAN_BRIGHT "From END worker: task %d ENDED %s" ANSI_RESET "\n", index, now_str().c_str());
}

std::vector<std::pair<int, int>> SegmentRasterizator::getDotsOfSegment()
{
    return dotsOfSegment;
}
\end{lstlisting}

\begin{lstlisting}[caption=Менеджер потоков,
label={list:director}]
Director::Director(std::queue<SegmentRasterizator> &startQueue_)
{
    startQueue = startQueue_;
}

void Director::processPrepare()
{
    int i = 0;
    for (SegmentRasterizator curSeg(startQueue.front()); startQueue.size();
         startQueue.pop(), curSeg = startQueue.front())
    {

        curSeg.prepareConstantsForRB(i++);
        middleQueue.push(curSeg);
    }
}

void Director::processRast()
{
    int i = 0;
    while (startQueue.size() || middleQueue.size())
    {
        if (middleQueue.empty())
            continue;
        SegmentRasterizator curSeg(middleQueue.front());

        curSeg.rastSegment(i++);

		endQueue.push(curSeg);
        middleQueue.pop();
    }
}

void Director::processCreate()
{
    int i = 0;
    while (startQueue.size() || middleQueue.size() || endQueue.size())
    {
        if (endQueue.empty())
            continue;
        SegmentRasterizator curSeg(endQueue.front());

        curSeg.createImg(i++);
        endQueue.pop();
        final.push_back(curSeg);
    }
}

void Director::initWork()
{
    workers[0] = std::thread(&Director::processPrepare, this);
    workers[1] = std::thread(&Director::processRast, this);
    workers[2] = std::thread(&Director::processCreate, this);

    workers[0].join();
    workers[1].join();
    workers[2].join();
}

std::vector<SegmentRasterizator> Director::getFinal() { return final; }
\end{lstlisting}

\section{Тестирование программного продукта}
В таблице~\ref{tabular:test_rec} приведены тесты для функций, реализующих алгоритм Брезенхема. Тесты пройдены успешно.

\begin{table}[h!]
	\begin{center}
	
	\caption{\label{tabular:test_rec} Тестирование функций}
		\begin{tabular}{c@{\hspace{7mm}}c@{\hspace{7mm}}c@{\hspace{7mm}}c@{\hspace{7mm}}c@{\hspace{7mm}}c@{\hspace{7mm}}}
			\hline
			Точка начала отрезка (x, y) & Точка конца отрезка (x, y) & Ожидаемый результат \\ \hline
			\vspace{4mm}
			 (1, 1)&
			 (3, 3)&
			 (1, 1), (2, 2), (3, 3)\\
			\vspace{2mm}
			\vspace{2mm}
			 (1, 1)&
			 (1, 3)&
			 (1, 1), (1, 2), (1, 3)\\
			\vspace{2mm}
			\vspace{2mm}
			 (1, 1)&
			 (2, 1)&
			 (1, 1), (2, 1)\\
			\vspace{2mm}
			\vspace{2mm}
			 (3, 3)&
			 (1, 1)&
			 (1, 1), (2, 2), (3, 3)\\
		\end{tabular}
	\end{center}
\end{table}
\newpage

\section*{Вывод}
Спроектированные алгоритмы были реализованы и протестированы.

\chapter{Исследовательская часть}
\section{Технические характеристики}
Технические характеристики ЭВМ, на котором выполнялись исследования:
\begin{itemize}
\item ОС: Manjaro Linux 20.1.1 Mikah;
\item Оперативная память: 16 Гб;
\item Процессор: Intel Core i7-10510U.
\end{itemize}

При проведении замеров времени ноутбук был подключен к сети электропитания.

\section{Пример работы программного обеспечения}
На рисунке \ref{img:example} приведен пример работы программы для 7 визуализируемых отрезков.

\begin{figure}
\begin{center}
\includegraphics[scale=0.9]{inc/img/example.png}
\captionsetup{justification=centering}
	\caption{Пример работы ПО.}
	\label{img:example}	
\end{center}
\end{figure}

\newpage
\section{Время выполнения алгоритмов}
Алгоритм тестировался на данных, сгенерированных случайным образом.

В таблице \ref{time1} предоставлено время работы над каждым отрезком в предоставленном примере каждого из выделенных этапов.

Из таблицы видно, что среднее время выполнения этапа 1 составляет $\approx 17428.6$ наносекунд. Среднее время выполнения этапа 2 составляет $\approx 38285.7$ наносекунд. Среднее время выполнения этапа 3 составляет $\approx 18571.4$ наносекунд. Таким образом, этап 1 сравним по среднему времени выполнения с этапом 2. Но после выполнения этапа 1 заметно, что последующие вызовы функции работают за константное время, равное 3000 наносекунд, что при наличии начальных "прогревочных" запусков вылилось бы в факт того, что этап 1 не был бы сопоставим по среднему времени выполнения с этапом 3. Этап 2 является самым долго выполняющимся.

\begin{table}[h]
	\begin{center}
		\caption{\label{time1} Замеры времени для выполнения выделенных этапов.}
		\begin{tabular}{|c |c |c |c|} 
 			\hline
 			&\multicolumn{3}{|c|}{Время обработки, нс}\\
 			\hline
			Номер отрезка & Этап 1 & Этап 2 & Этап 3\\ [0.5ex] 
 			\hline\hline
 			0 & 97000 & 74000 & 27000 \\
 			\hline
 			1 & 10000 & 38000 & 21000 \\
 			\hline
			2 & 3000 & 32000 & 16000 \\
			\hline
			3 & 3000 & 35000 & 19000 \\
			\hline
			4 & 3000 & 22000 & 12000 \\
			\hline
			5 & 3000 & 35000 & 16000 \\
			\hline
			6 & 3000 & 32000 & 19000 \\
			\hline
			\end{tabular}
	\end{center}
\end{table}

\newpage

\section*{Вывод}
При сравнении результатов замеров по времени стало известно, что самым быстрым этапом конвейера оказался этап 1. При этом, самым медленным из трех рассмотренных - этап 2.

В среднем этап 1 работает быстрее этапа 2 на $\approx 20857.1$ наносекунд. При этом, при четвертой обработке отрезка разница в скорости выполнения составила 32000 наносекунд.

Этап 3 в среднем работает быстрее этапа 2 на $\approx 19714.3$ наносекунд. При этом, при шестой обработке отрезка разница в скорости выполнения составила 19000 наносекунд.

Таким образом, среднее время выполнения алгоритма для каждого отрезка составило $\approx 74285.71$ наносекунд.

\chapter*{Заключение}
\addcontentsline{toc}{chapter}{Заключение}
В ходе выполнения лабораторной работы была выполнена цель и следующие задачи:
\begin{enumerate}
\item[1)] было изучено асинхронное взаимодействие на примере конвейерной обработки данных;
\item[2)] была спроектирована система конвейерных вычислений;
\item[3)] была реализована система конвейерных вычислений;
\item[4)] была протестирована реализованная система;
\item[5)] был подготовлен отчёт по проведенной работе.
\end{enumerate}

Исследования показали, что в среднем:
\begin{enumerate}
\item[1)] этап 1 работает быстрее этапа 2 на $\approx 20857.1$ наносекунд;
\item[2)] этап 3 работает быстрее этапа 2 на $\approx 19714.3$ наносекунд;
\item[3)] среднее время выполнения алгоритма составило $\approx 74285.71$ наносекунд.
\end{enumerate}

\addcontentsline{toc}{chapter}{Литература}
\bibliographystyle{utf8gost705u}  % стилевой файл для оформления по ГОСТу
\bibliography{biblio.bib}          % имя библиографической базы (bib-файла)


\end{document} 
