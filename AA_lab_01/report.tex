\documentclass[12pt]{report}
\usepackage[utf8]{inputenc}
\usepackage[russian]{babel}
%\usepackage[14pt]{extsizes}
\usepackage{listings}

% Для листинга кода:
\lstset{ %
language=c++,                 % выбор языка для подсветки (здесь это С)
basicstyle=\small\sffamily, % размер и начертание шрифта для подсветки кода
numbers=left,               % где поставить нумерацию строк (слева\справа)
numberstyle=\tiny,           % размер шрифта для номеров строк
stepnumber=1,                   % размер шага между двумя номерами строк
numbersep=5pt,                % как далеко отстоят номера строк от подсвечиваемого кода
showspaces=false,            % показывать или нет пробелы специальными отступами
showstringspaces=false,      % показывать или нет пробелы в строках
showtabs=false,             % показывать или нет табуляцию в строках
frame=single,              % рисовать рамку вокруг кода
tabsize=2,                 % размер табуляции по умолчанию равен 2 пробелам
captionpos=t,              % позиция заголовка вверху [t] или внизу [b] 
breaklines=true,           % автоматически переносить строки (да\нет)
breakatwhitespace=false, % переносить строки только если есть пробел
escapeinside={\#*}{*)}   % если нужно добавить комментарии в коде
}

\usepackage{hyperref}
\hypersetup{
    linktoc=all,     %set to all if you want both sections and subsections linked
    linkcolor=blue,  %choose some color if you want links to stand out
}

% Для измененных титулов глав:
\usepackage{titlesec, blindtext, color} % подключаем нужные пакеты
\definecolor{gray75}{gray}{0.75} % определяем цвет
\newcommand{\hsp}{\hspace{20pt}} % длина линии в 20pt
% titleformat определяет стиль
\titleformat{\chapter}[hang]{\Huge\bfseries}{\thechapter\hsp\textcolor{gray75}{|}\hsp}{0pt}{\Huge\bfseries}


% plot
\usepackage{pgfplots}
\usepackage{filecontents}
\usetikzlibrary{datavisualization}
\usetikzlibrary{datavisualization.formats.functions}
\begin{filecontents}{LevR.dat}
3 0.00006
4 0.00033
5 0.00033
6 0.00780
7 0.03876
8 0.20780
9 1.18171
\end{filecontents}

\begin{filecontents}{LevT.dat}
3 0.00003
4 0.00003
5 0.00005
6 0.00005
7 0.00007
8 0.00008
9 0.00009
\end{filecontents}

\begin{filecontents}{DamLevR.dat}
3 0.00006
4 0.00027
5 0.00143
6 0.00787
7 0.04130
8 0.23259
9 1.26665
\end{filecontents}

\begin{filecontents}{DamLevT.dat}
3 0.00003
4 0.00003
5 0.00005
6 0.00006
7 0.00007
8 0.00013
9 0.00012
\end{filecontents}


\begin{document}
%\def\chaptername{} % убирает "Глава"
\begin{titlepage}
	\fontsize{12pt}{12pt}\selectfont
	\noindent \begin{minipage}{0.15\textwidth}
		\includegraphics[width=\linewidth]{inc/img/b_logo.jpg}
	\end{minipage}
	\noindent\begin{minipage}{0.9\textwidth}\centering
		\textbf{Министерство науки и высшего образования Российской Федерации}\\
		\textbf{Федеральное государственное бюджетное образовательное учреждение высшего образования}\\
		\textbf{«Московский государственный технический университет имени Н.Э.~Баумана}\\
		\textbf{(национальный исследовательский университет)»}\\
		\textbf{(МГТУ им. Н.Э.~Баумана)}
	\end{minipage}
	
	\noindent\rule{15cm}{3pt}
	\newline\newline
	\noindent ФАКУЛЬТЕТ \underline{~~~~~~~~~~~~~~~~«Информатика и системы управления»~~~~~~~~~~~~~~~~} \newline\newline
	\noindent КАФЕДРА \underline{«Программное обеспечение ЭВМ и информационные технологии»}\newline\newline\newline\newline\newline\newline\newline
	
	
	\begin{center}
		\Large\textbf{Отчет по лабораторной работе №1 по курсу "Анализ алгоритмов"}\newline
	\end{center}
	
	\noindent\textbf{Тема} \underline{Расстояния Левенштейна и Дамерау-Левенштейна}\newline\newline\newline
	\noindent\textbf{Студент} \underline{Якуба Д. В.}\newline\newline
	\noindent\textbf{Группа} \underline{ИУ7-53Б}\newline\newline
	\noindent\textbf{Оценка (баллы)} \underline{~~~~~~~~~~~~~~~~~~~}\newline\newline
	\noindent\textbf{Преподаватели} \underline{Волкова Л.Л., Строганов Ю.В.}\newline
	
	\begin{center}
		\vfill
		Москва~---~\the\year
		~г.
	\end{center}
\end{titlepage}

\tableofcontents

\newpage
\chapter*{Введение}
\addcontentsline{toc}{chapter}{Введение}
\section*{Цель лабораторной работы}
Изучение и реализация алгоритмов нахождения расстояния Левенштейна и Дамерау-Левенштейна.
\section*{Определение}
Расстояние Левенштейна - это мера, определяющая различие двух последовательностей символов. По неформальному определению расстояние Левенштейна между двумя словами - это мимнимальное количество односимвольных изменений (вставок, удалений или замен), необходимых для преобразования одного слова в другое.

Расстояние Левенштейна применяется в теории информации и компьютерной лингвистике для:

\begin{itemize}
	\item Авто-исправления ошибок в слове
	\item Сравнение введёных строк со словарными в поисковых запросах
	\item Сравнения текстовых файлов утилитой diff
	\item В биоинформатике для сравнения генов, хромосом и белков
\end{itemize}

Расстояние Дамерау-Левенштейна - это также мера, определяющая различие двух последовательностей символов, однако набор доступных операций для преобразований строк расширяется - добавляется операция транспозиции (перестановка двух соседствующих символов).

\section*{Задачи лабораторной работы}
Задачами данной лабораторной являются:
\begin{enumerate}
  	\item Изучение алгоритмов Левенштейна и Дамерау-Левенштейна;
	\item Применение метода динамического программирования для реализации указанных алгоритмов; 
	\item Получение практических навыков реализации алгоритмов Левенштейна и Дамерау-Левенштейна; 
	\item Сравнительный анализ реализованных алгоритмов; 
	\item Подготовка отчёта по проведённой работе. 
\end{enumerate}


\chapter{Аналитическая часть}
Расстояние Левенштейна - это минимальное ннобходимое количество операций (вставки, замены, удаления) редакторских операций для преобразования одной строки в другую.

При нахождении расстояния Дамерау — Левенштейна добавляется операция транспозиции (перестановки соседствующих символов).  
 
\textbf{Обозначение редакторских операций:} 
\begin{enumerate}
	\item I - вставка символа;
	\item R - замена символа;
	\item D - удаление символа;
	\item M - бездействие (совпадение символов);
\end{enumerate}

При этом для каждой операции задаётся своя цена (или штраф). Для решения задачи необходимо найти последовательность операций, минимизирующую суммарную цену всех проведённых операций. При этом следует отметить, что:
\begin{enumerate}
	\item $price(x, x) = 0$ - цена замены символа $x$ самого на себя;
	\item $price(x, y) = 0$   $(x \neq y)$ - цена замены символа $x$ на символ $y$;
	\item $price(\alpha, x)$ - цена вставки символа $x$;
	\item $price(x, \alpha)$ - цена удаления символа $x$.
\end{enumerate}

\section{Рекурсивный алгоритм нахождения расстояния Левенштейна}
Пусть $s_{1}$ и $s_{2}$ — две некоторые строки. Тогда расстояние Левенштейна может быть вычислено по формуле \ref{recAlg}

\begin{equation}
\label{recAlg}
\begin{displaymath}
D(i,j) = \left\{ \begin{array}{ll}
 0 & \textrm{$i = 0, j = 0$}\\
 i & \textrm{$j = 0, i > 0$}\\
 j & \textrm{$i = 0, j > 0$}\\
min(\\
D(i,j-1)+1\\
D(i-1, j) +1 &\textrm{$j>0, i>0$}\\
D(i-1, j-1) + m(s_{1}[i], s_{2}[j])\\
)
  \end{array} \right.
  \end{equation}
\end{displaymath}
где функция $m(x,y)$ ($x$ и $y$ - символы) равна нулю, если $a=b$, и единице в противном случае; $x[i]$ - это $i$-ый символ строки $x$.

При этом очевидны следующие факты:
\begin{enumerate}
	\item $D(s_{1}, s_{2})$ \geq $ ||s_{1}| - |s_{2}||$
	\item $D(s_{1}, s_{2})$ \leq $ max(|s_{1}|, |s_{2}|)$
	\item $D(s_{1}, s_{2}) = 0$ \Leftrightarrow $ s_{1} = s_{2}$
\end{enumerate}

Суть рекурсивного алгоритма заключается в реализации формулы \ref{recAlg}.

\section{Рекурсивный алгоритм нахождения расстояния Левенштейна с использованием матрицы}
Для оптимизации рекурсивного алгоритма нахождения расстояния Левенштейна допустимо добавить матрицу для хранения значений $D(i, j)$ для того, чтобы не вычислять их заново раз за разом. Таким образом, при обработке ещё не затронутых данных, результат нахождения расстояния будет занесён в матрицу расстояний. В ином случае, если для рассматриваемого случая информация о расстоянии уже присутствует в матрице, алгоритм будет переходить к следующему шагу.

\section{Итеративный алгоритм нахождения расстояния Левенштейна с использованием матрицы}
Данный алгоритм также заключается в решении задачи с использованием матрицы расстояний. От уже рассмотренного рекурсивного алгоритма нахождения расстояния Левенштейна с использованием матрицы итеративный алгоритм заключается в построчном заполнении матрицы последовательно вычисляемыми $D(i, j)$.

\section{Алгоритм нахождения расстояния Дамерау-Левенштейна с использованием матрицы}

Расстояние Дамерау-Левенштейна вычисляется по следующей рекуррентной формуле:

\begin{displaymath}\\
\begin{equation}
\label{DLalg}
D(i,j) = \left\{ \begin{array}{ll}
 0, & \textrm{$i = 0, j = 0$}\\
 i, & \textrm{$j = 0, i > 0$}\\
 j, & \textrm{$i = 0, j > 0$}\\
min(\\
D(i,j-1)+1,\\
D(i-1, j) +1, &\textrm{$j>0, i>0$}\\
D(i-1, j-1) + m(s_{1}[i], s_{2}[j])\\
D(i-2, j-2) + 1, &\textrm{$i,j>1$, $s_{1}[i] = s_{2}[j-1],s_{1}[i-1]=s_{2}[j] $}\\
)
  \end{array} \right.
  \end{equation}
\end{displaymath}

Применение данной формулы для реализации рекурсивного алгоритма при больших значениях $i, j$ будет работать достаточно долго по тем же причинам, что и реализация рекурсивного алгоритма поиска расстояния Левенштейна. По этой причине целесообразно ввести матрицу, в которой будут храниться вычисленные по формуле значения.

\section{Вывод}
Для каждого рассмотренного алгоритма имеется некоторая рекуррентная формула, что даёт возможность изучить как рекурсивные, так и итеративные реализации алгоритмов. Для оптимизации рекурсивных алгоритмов в рассмотрение вводится матрица, в которую записываются все промежуточные вычисленные значения. Эта же матрица применяется и при реализации итеративных алгоритмов.

\newpage

\chapter{Конструкторская часть}
\section{Блок-схема рекурсивного алгоритма Левенштейна}
\includegraphics[width=\linewidth]{inc/img/LevRec.jpg}{\newline \textbf{Лох}}
\section{Блок-схема рекурсивного алгоритма Левенштейна с использованием матрицы}
\includegraphics[width=\linewidth]{inc/img/LevMatrRec.jpg}{...}
\section{Блок-схема итеративного алгоритма Левенштейна}
\includegraphics[width=\linewidth]{inc/img/LevIter.jpg}{...}
\section{Блок-схема алгоритма Дамерау-Левенштейна}
\includegraphics[width=\linewidth]{inc/img/DamLev.jpg}{...}

\chapter{Технологическая часть}
\section{Требования к программному обеспечению}
\begin{itemize}
\item Входные данные - две строки в любой раскладке;
\item Выходные данные - искомое расстояние для выбранного метода и матрица расстояний для методов с её использованием.
\end{itemize}
\section{Средства реализации программного обеспечения}
При написании программного продукта был использован язык программирования C++.

Данный выбор обусловлен следующими факторами:
\begin{itemize}
\item Данный язык программирования преподавался в рамках курса объектно-ориентированного программирования;
\item Высокая вычислительная производительность;
\item Большое количество справочной литературы.
\end{itemize}

Для тестирования производительности реализаций алгоритмов использовалась утилита QueryPerfomanceCounter, объявленная в заголовочном файле "windows.h".

При написаннии программного продукта использовалась среда разработки QT Creator.

Данный выбор обусловлен следующими факторами:
\begin{itemize}
\item Основы работы с данной средой разработки преподавался в рамках курса программирования на Си;
\item QT Creator позволяет работать с расширением QtDesign, которое позволяет создать удобный интерфейс для программного продукта в сжатые сроки.
\end{itemize}

\section{Сведения о модулях программы}
Программа состоит из:
\begin{itemize}
	\item main.py - главный файл программы, в котором располагаются алгоритмы и меню
	\item test.py - файл с тестами 
\end{itemize}

\begin{lstlisting}[label=some-code,caption=Функция реализации рекурсивного алгоритма Левенштейна]
size_t MainWindow::levenshteinRecursive(QString fWord, QString sWord)
{
    if (!fWord.size() || !sWord.size())
        return fWord.size() + sWord.size();

    return std::min({levenshteinRecursive(fWord, sWord.mid(0, sWord.size() - 1)) + 1,
    levenshteinRecursive(fWord.mid(0, fWord.size() - 1), sWord) + 1,
    levenshteinRecursive(fWord.mid(0, fWord.size() - 1), sWord.mid(0, sWord.size() - 1)) +
    ((fWord.back() == sWord.back()) ? 0 : 1)});
}
\end{lstlisting}

\begin{lstlisting}[label=some-code,caption=Функция реализации рекурсивного алгоритма Левенштейна с использованием матрицы расстояний]
size_t MainWindow::levenshteinRecursiveMatrix(
QString fWord, QString sWord, std::vector<std::vector<int>> &matrix)
{
    //    qDebug() << "Current Words are: " << fWord << sWord;
    if (matrix[sWord.size()][fWord.size()] != std::numeric_limits<int>().max())
        return matrix[sWord.size()][fWord.size()];

    matrix[sWord.size()][fWord.size()] =
    std::min({levenshteinRecursiveMatrix(fWord.mid(0, fWord.size() - 1), sWord, matrix) + 1,
    levenshteinRecursiveMatrix(fWord, sWord.mid(0, sWord.size() - 1), matrix) + 1,
    levenshteinRecursiveMatrix(
    fWord.mid(0, fWord.size() - 1), sWord.mid(0, sWord.size() - 1), matrix) +
    ((fWord.back() == sWord.back()) ? 0 : 1)});

    return matrix[sWord.size()][fWord.size()];
}
\end{lstlisting}

\begin{lstlisting}[label=some-code,caption=Функция реализации итеративного алгоритма Левенштейна]
size_t MainWindow::levenshteinNonRecursiveMatrix(
QString fWord, QString sWord, std::vector<std::vector<int>> &matrix)
{
    for (int i = 1; i <= sWord.size(); i++)
        for (int j = 1; j <= fWord.size(); j++)
            matrix[i][j] = std::min({matrix[i - 1][j] + 1, matrix[i][j - 1] + 1,
            matrix[i - 1][j - 1] +
            (((fWord.mid(0, j)).back() == sWord.mid(0, i).back()) ? 0 : 1)});

    return matrix[sWord.size()][fWord.size()];
}
\end{lstlisting}

\begin{lstlisting}[label=some-code,caption=Функция реализации алгоритма Дамерау-Левенштейна]
bool canBeTranspose(QString fStr, QString sStr, size_t i, size_t j)
{
    return fStr.at(j - 1) == sStr.at(i - 2) && fStr.at(j - 2) == sStr.at(i - 1);
}

size_t MainWindow::damerauLev(
QString fWord, QString sWord, std::vector<std::vector<int>> &matrix)
{
    for (int i = 1; i <= sWord.size(); i++)
        for (int j = 1; j <= fWord.size(); j++)
        {
            int temp = std::min({matrix[i - 1][j] + 1, matrix[i][j - 1] + 1,
            matrix[i - 1][j - 1] +
            (((fWord.mid(0, j)).back() == sWord.mid(0, i).back()) ? 0 : 1)});
            if (i > 1 && j > 1 && canBeTranspose(fWord, sWord, i, j))
                temp = std::min(temp, matrix[i - 2][j - 2] + 1);
            matrix[i][j] = temp;
        }
    return matrix[sWord.size()][fWord.size()];
}
\end{lstlisting}

\section{Тестирование программного продукта}
В таблице \ref{table} приведены тестовые данные и вывод программы для алгоритмов вычисления расстояния Левенштейна и Дамерау-Левенштейна. Тесты пройдены успешно.

\label{table}
\begin{table}[h]
	\begin{center}
		\caption{\label{tabular:functional_test} Тесты}
		\begin{tabular}{|c|c|c|c|}
	\hline
			                    &                    & \multicolumn{2}{c|}{\bfseries Ожидаемый результат}    \\ \cline{3-4}\hline
	Строка 1& Строка 2 & Алг. Левенштейна & Алг. Дамерау-Левенштейна \\ [0.5ex] 
 	\hline\hline
 	a & a & a & a\\
 	\hline
 	a & a & a & a\\
 	\hline
	a & a & 0a & a\\
	\hline
	a & a & a & a\\
	\hline
	0a & a & a & a\\
	\hline
	a & a & a & a\\
	\hline
	a & a & a & a\\
	\hline
		\end{tabular}
	\end{center}
\end{table}

\section{Вывод}
Новый автомат...

\chapter{Исследовательская часть}

\section{Пример работы программного обеспечения}
Ниже на рисунках (АШРЕФЫ ПИХАЙ КОМУ ГОВОРЮ) предоставлены таблицы, сгенерированные по окончанию работы каждого из алгоритмов.

\section{Технические характеристики}
Технические характеристики ЭВМ, на котором выполнялись исследования:
\begin{itemize}
\item бла-бла-бла
\end{itemize}

\section{Время выполнения алгоритмов}
Ога...

\section{Оценка затрат памяти}
Ого как умно сформулировал...

\section{Вывод}
Ё-моё...

\begin{center}
	\begin{tabular}{|c c c c c|} 
 	\hline
	len & Lev(R) & DamLev(R) & Lev(T) & DamLev(T) \\ [0.5ex] 
 	\hline\hline
 	3 & 0.00006 & 0.00006 & 0.00003 & 0.00003\\
 	\hline
 	4 & 0.00033 & 0.00027 & 0.00003 & 0.00003\\
 	\hline
	5 & 0.00141 & 0.00143 & 0.00005 & 0.00005\\
	\hline
	6 & 0.00780 & 0.00787 & 0.00005 & 0.00006\\
	\hline
	7 & 0.03876 & 0.04130 & 0.00007 & 0.00007\\
	\hline
	8 & 0.20780 & 0.23259 & 0.00008 & 0.00013\\
	\hline
	9 & 1.18171 & 1.26665 & 0.00009 & 0.00012\\
	\hline
	\end{tabular}
\end{center}





\begin{tikzpicture}

\begin{axis}[
    	axis lines = left,
    	xlabel = $len$,
    	ylabel = {$time$},
	legend pos=north west,
	ymajorgrids=true
]
\addplot[color=red] table[x index=0, y index=1] {LevR.dat}; 
\addplot[color=orange] table[x index=0, y index=1] {DamLevR.dat};
\addplot[color=blue, mark=square] table[x index=0, y index=1] {LevT.dat};
\addplot[color=green, mark=square] table[x index=0, y index=1] {DamLevT.dat};

\addlegendentry{LevR}
\addlegendentry{DamLevR}
\addlegendentry{LevT}
\addlegendentry{DamLevT}
\end{axis}
\end{tikzpicture}


\par
Рекурсивные реализации сравнимы по времени между собой. При увеличении длины строк становится очевидна выигрышность по времени матричного варианта. Уже при длине в 9 символов матричная реализация в 10,000 раз быстрее.

\chapter*{Заключение}
\addcontentsline{toc}{chapter}{Заключение}
Был изучен метод динамического программирования на материале алгоритмов Левенштейна и Дамерау-Левенштейна.
Также изучены алгоритмы Левенштейна и Дамерау-Левенштейна нахождения расстояния между строками, получены практические навыки раелизации указанных алгоритмов
в матричной  и рекурсивных версиях. 

Экспериментально было подтверждено различие во временной эффективности рекурсивной и нерекурсивной реализаций выбранного алгоритма определения расстояния между строками при помощи разработаного программного обеспечения на материале замеров процессорного времени выполнения реализации на варьирующихся длинах строк. 

В результате исследований пришла к выводу, что матричная реализация данных алгоритмов заметно выигрывает по времени при росте длины строк.


\end{document} 
